\documentclass{article}
\usepackage{amsmath}
\usepackage{amssymb}
\usepackage{graphicx}
\usepackage{listings}
\usepackage{xcolor}
\usepackage{geometry}
\geometry{a4paper, margin=1in}

\title{Raport z Symulacji Monte Carlo}
\author{}
\date{}

\begin{document}

\maketitle

\section*{Opis wyników}

W ramach eksperymentu wykorzystano metodę Monte Carlo do przybliżonego obliczenia wartości całek oznaczonych dla następujących funkcji:
\begin{itemize}
    \item \( f(x) = \sin(x) \) na przedziale \( [0, \pi] \),
    \item \( g(x) = 4x(1 - x)^3 \) na przedziale \( [0, 1] \),
    \item \( h(x) = x^{1/3} \) na przedziale \( [0, 8] \).
\end{itemize}

Dla każdej z tych funkcji wykonano symulacje przy różnych liczbach próbek (\( n \)) z zakresu od 50 do 5000, a także przeprowadzono 5 i 50 powtórzeń dla każdej liczby próbek. Wyniki zostały zwizualizowane na wykresach w celu porównania przybliżonych wartości całek Monte Carlo z ich rzeczywistymi wartościami:
\begin{itemize}
    \item \( \approx 2 \) dla \( f(x) = \sin(x) \),
    \item \( \approx 0.2 \) dla \( g(x) = 4x(1 - x)^3 \),
    \item \( \approx 12 \) dla \( h(x) = x^{1/3} \).
\end{itemize}

Na wykresach przedstawiono wyniki pojedynczych symulacji oraz ich średnie wartości, umożliwiając porównanie przybliżonych wyników Monte Carlo z wartościami teoretycznymi.

\section*{Wnioski}

\begin{enumerate}
    \item \textbf{Dokładność przybliżeń}: \\
    Metoda Monte Carlo skutecznie przybliża wartości całek dla wszystkich analizowanych funkcji. Wraz ze wzrostem liczby próbek \( n \) rośnie stabilność wyników, co pozwala uzyskać przybliżenia bliższe wartości rzeczywistej.
    
    \item \textbf{Wpływ liczby powtórzeń}: \\
    Zwiększenie liczby powtórzeń (do 50) poprawia stabilność wyników, zmniejszając odchylenia od wartości rzeczywistej. Pokazuje to, że liczba powtórzeń jest kluczowa dla dokładności wyników przy mniejszych wartościach \( n \).
    
    \item \textbf{Charakterystyka funkcji a dokładność}: \\
    Funkcje o bardziej złożonym kształcie, takie jak \( f(x) = \sin(x) \) i \( g(x) = 4x(1 - x)^3 \), wymagają większej liczby próbek, aby uzyskać zadowalające przybliżenia, w porównaniu do funkcji \( h(x) = x^{1/3} \).
    
    \item \textbf{Efektywność obliczeniowa}: \\
    Choć metoda Monte Carlo jest łatwa w implementacji, czas wykonania rośnie wraz ze wzrostem liczby próbek i powtórzeń. W praktyce liczba próbek i powtórzeń powinna być dostosowana do wymaganej dokładności oraz dostępnych zasobów obliczeniowych.
\end{enumerate}

\section*{Podsumowanie}

Symulacja Monte Carlo pozwala uzyskać wartości bliskie rzeczywistym dla wybranych funkcji, stanowiąc skuteczne narzędzie w przypadkach, gdy analityczne obliczenia są trudne. Zwiększenie liczby próbek i powtórzeń znacząco poprawia dokładność wyników, potwierdzając efektywność tej metody.

\end{document}
