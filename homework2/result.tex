\documentclass{article}
\usepackage{amsmath}
\usepackage{amssymb}
\usepackage{graphicx}
\usepackage{listings}
\usepackage{xcolor}
\usepackage{geometry}
\geometry{a4paper, margin=1in}

\title{Raport z Symulacji Urn}
\author{Jan Ryszkiewicz}
\date{}

\begin{document}

\maketitle

\section*{Opis wyników}

W ramach zadania przeprowadziłem symulację wrzucana kul do n urn po 50 razy śledząc następujące informacje:
    \item \( B \) moment pierwszej kolizji,
    \item \( C \) minimalna liczba rzutów, po której w kazdej z urn jest co najmniej jedna kula,
    \item \( D \) minimalna liczba rzutów, po której w kazdej z urn są co najmniej dwie kule,
    \item \( U \) liczba pustych urn po wrzuceniu n kul.
    \item \( D - C \) liczba rzutów od momentu Cn potrzebna do tego, zeby w każdej urnie były conajmniej dwie kule.
\end{itemize}

Na wykresach (pliki .png) przedstawiam wyniki pojedynczych symulacji oraz ich średnie wartości dla danych n.

\section*{Wnioski}

\begin{enumerate}
    \item \textbf{Dokładność przybliżeń}: \\
    Metoda Monte Carlo skutecznie przybliża wartości całek dla wszystkich analizowanych funkcji. Wraz ze wzrostem liczby próbek \( n \) rośnie stabilność wyników, co pozwala uzyskać przybliżenia bliższe wartości rzeczywistej.
    
    \item \textbf{Wpływ liczby powtórzeń}: \\
    Zwiększenie liczby powtórzeń poprawia stabilność wyników, zmniejszając odchylenia od wartości rzeczywistej. Widać przez to, że liczba powtórzeń jest kluczowa dla dokładności wyników.
    
    \item \textbf{Charakterystyka zwiększania ilości powtórzeń dla \( n\)}: \\
    Warto zauważyć że dla dużych wartości \( n\) wynik zaczylają oscylować wokół wartości faktycznej w stałym zakresie co obrazuje że zwiększanie liczby powtórzeń nie daje lepszego przybliżenia oraz że lepszym sposobem jest branie średniej z tych wyników.
    
    \item \textbf{Efektywność obliczeniowa}: \\
    Choć metoda Monte Carlo jest prosta w implementacji, wymaga dużej mocy obliczeniowej dlatego powinna być używana by w relatywnie dokładny sposób przybliżać wartości o średniej dokładności.
\end{enumerate}

\section*{Podsumowanie}

Symulacja Monte Carlo pozwala uzyskać wartości bliskie rzeczywistym dla wybranych funkcji, stanowiąc skuteczne narzędzie w przypadkach, gdy analityczne obliczenia są trudne. Zwiększenie liczby próbek i powtórzeń znacząco poprawia dokładność wyników, potwierdzając efektywność tej metody.
\vspace{1cm} 

\hfill
\textbf{Jan Ryszkiewicz}

\end{document}
