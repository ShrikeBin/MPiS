\documentclass{article}
\usepackage{graphicx}
\usepackage{amsmath}
\usepackage{pgfplots}
\usepackage{tikz}

\title{Sprawozdanie z realizacji zadań 1-3}
\author{Jan Ryszkiewicz}
\date{\today}

\begin{document}

\maketitle

\section*{Zadanie 1: The Power of Two Choices / balanced allocation}

W zadaniu 1 rozszerzyliśmy symulację z poprzedniego zadania domowego o eksperymentalne wyznaczenie maksymalnej liczby kul w urnie po wrzuceniu \(n\) kul do \(n\) urn. Testowaliśmy dwa przypadki:

\begin{itemize}
    \item (a) Dla każdej kuli wybieramy niezależnie i jednostajnie losowo jedną z \(n\) urn, w której umieszczamy kulę.
    \item (b) Dla każdej kuli wybieramy niezależnie i jednostajnie losowo dwie urny, a kulę umieszczamy w najmniej zapełnionej z wybranych urn.
\end{itemize}

Po przeprowadzeniu symulacji dla \(n \in \{10 000, 20 000, \dots, 1 000 000\}\) oraz obliczeniu średniego maksymalnego zapełnienia \(L^{(d)}_n\) dla obu przypadków (gdzie \(d = 1\) i \(d = 2\)), uzyskano wyniki, które zaprezentowano na poniższych wykresach. Dodatkowo, wykresy przedstawiają funkcje \(f_1(n) = \frac{\ln n}{\ln \ln n}\) oraz \(f_2(n) = \frac{\ln \ln n}{\ln 2}\).

\begin{figure}[ht]
    \centering
    \includegraphics[width=\textwidth]{wykres1.png}
    \caption{Wykresy \(L^{(1)}_n\) i \(L^{(2)}_n\) oraz funkcji \(f_1(n)\) i \(f_2(n)\) w zależności od \(n\).}
    \label{fig:wykres1}
\end{figure}

Na podstawie wykresów stwierdzono, że rozkład wyników dla obydwu przypadków jest skoncentrowany wokół wartości średniej, jednak w przypadku \(d = 2\) rozkład jest bardziej spłaszczony, co sugeruje mniejsze zróżnicowanie wyników. Asymptotycznie wartości \(L^{(d)}_n\) przybliżają się do wartości \(\frac{\ln n}{\ln \ln n}\) dla \(d = 1\) oraz \(\frac{\ln \ln n}{\ln 2}\) dla \(d = 2\), co potwierdza stosowanie skal logarytmicznych w wykresach.

\section*{Zadanie 2: Sortowanie przez wstawianie losowych danych}

W zadaniu 2 zaimplementowaliśmy algorytm sortowania przez wstawianie (INSERTIONSORT) i przeprowadziliśmy eksperymenty dla \(n \in \{100, 200, \dots, 10 000\}\) oraz \(k = 50\) powtórzeń dla każdego \(n\). Dla każdego eksperymentu zebrano dane dotyczące liczby wykonanych porównań oraz przestawień kluczy.

Na podstawie zebranych danych przedstawiono poniższe wykresy.

\begin{figure}[ht]
    \centering
    \includegraphics[width=\textwidth]{wykres2a.png}
    \caption{Średnia liczba porównań jako funkcja \(n\).}
    \label{fig:wykres2a}
\end{figure}

\begin{figure}[ht]
    \centering
    \includegraphics[width=\textwidth]{wykres2b.png}
    \caption{Średnia liczba przestawień jako funkcja \(n\).}
    \label{fig:wykres2b}
\end{figure}

\begin{figure}[ht]
    \centering
    \includegraphics[width=\textwidth]{wykres2c.png}
    \caption{Iloraz \(\frac{cmp(n)}{n}\) oraz \(\frac{cmp(n)}{n^2}\) jako funkcje \(n\).}
    \label{fig:wykres2c}
\end{figure}

\begin{figure}[ht]
    \centering
    \includegraphics[width=\textwidth]{wykres2d.png}
    \caption{Iloraz \(\frac{s(n)}{n}\) oraz \(\frac{s(n)}{n^2}\) jako funkcje \(n\).}
    \label{fig:wykres2d}
\end{figure}

Z wykresów wynika, że liczba porównań rośnie niemal kwadratowo w zależności od \(n\), natomiast liczba przestawień rośnie w sposób bardziej liniowy. Wartości ilorazów \(\frac{cmp(n)}{n}\) oraz \(\frac{cmp(n)}{n^2}\) wskazują na kwadratową złożoność czasową algorytmu w przypadku sortowania przez wstawianie.

\section*{Zadanie 3: Uproszczony model komunikacji z zakłóceniami}

W zadaniu 3 przeprowadzono symulacje w celu eksperymentalnego zbadania minimalnej liczby rund \(T_n\) potrzebnej do rozesłania informacji w sieci o topologii gwiazdy z zakłóceniami. Eksperymenty przeprowadzono dla \(p = 0.5\) oraz \(p = 0.1\).

Na podstawie uzyskanych wyników przedstawiono wykresy liczby rund potrzebnych do rozesłania informacji.

\begin{figure}[ht]
    \centering
    \includegraphics[width=\textwidth]{wykres3a.png}
    \caption{Średnia liczba rund \(T_n\) jako funkcja \(n\) dla \(p = 0.5\).}
    \label{fig:wykres3a}
\end{figure}

\begin{figure}[ht]
    \centering
    \includegraphics[width=\textwidth]{wykres3b.png}
    \caption{Średnia liczba rund \(T_n\) jako funkcja \(n\) dla \(p = 0.1\).}
    \label{fig:wykres3b}
\end{figure}

Z wykresów wynika, że liczba rund potrzebnych do rozesłania informacji rośnie wraz z \(n\), ale dla mniejszego prawdopodobieństwa \(p\) liczba rund rośnie szybciej. Potwierdza to, że mniejsze prawdopodobieństwo odbioru informacji wydłuża czas potrzebny do jej rozesłania w sieci.

\section*{Wnioski}

W ramach przeprowadzonych eksperymentów uzyskano wyniki wskazujące na różne zachowania algorytmów w zależności od zmieniających się parametrów. W zadaniu 1 zauważono asymptotyczne podejście do wartości średnich, a w zadaniu 2 potwierdzono złożoność kwadratową algorytmu INSERTIONSORT. W zadaniu 3 zaobserwowano wpływ zmieniającego się prawdopodobieństwa na czas rozesłania informacji w sieci.

\end{document}
